\documentclass[UTF8]{ctexart} %支持utf-8
\setlength{\headheight}{15pt} %主标题15pt大

\usepackage[a4paper, left = 3.18cm, right = 3.18cm, top = 2.54cm, bottom = 2.54cm]{geometry}
\usepackage{ulem}
\usepackage{color}

\usepackage{fancyhdr}
\pagestyle{fancy}
\fancyhf{}
\lhead{202302 古诗文初中组 初试题目}
\rhead{ 第\thepage 页 \quad 共 \pageref{lastpage} 页 }
% 页眉标注主题和页码

\title{202302 古诗文初中组 初试题目}
\date{\today}
% 标题

\begin{document}
\maketitle
\newpage
\setcounter{page}{1}
\section{选择题}
\paragraph{
1. 古代著名书法家,有“书圣”之称的是(\color{red}A\color{black})。 \\
    A. 王羲之 \\
    B. 颜真卿 \\
    C. 欧阳询 \\
    D. 怀素
}
\paragraph{
2. “羲之欣然写毕,笼鹅而归”中,“笼”字的用法是(\color{red}A\color{black})。 \\
    A. 名词作动词 \\
    B. 动词作名词 \\
    C. 名词作形容词 \\
    D. 形容词作动词
}
\paragraph{
3. 下列古代名人与其爱好物不相符合的是(\color{red}C\color{black})。 \\
    A. 王羲之——鹅 \\
    B. 陶渊明——菊 \\
    C. 郑板桥——梅 \\
    D. 周敦颐——莲
}
\paragraph{
4. “其母引刀裂其织,以此诫之”中的“引”应解释为(\color{red}B\color{black})。 \\
    A. 牵引 \\
    B. 拿起 \\
    C. 拉开 \\
    D. 导引
}
\paragraph{
5. 下列著作与编撰者对应不正确的是(\color{red}B\color{black})。 \\
    A. 《韩诗外传》——韩婴 \\
    B. 《战国策》——刘安 \\
    C. 《礼记》——戴圣 \\
    D. 《梦溪笔谈》——沈括
}
\paragraph{
6. 下列句中“之”字用法与其他三项不同的是(\color{red}D\color{black})。 \\
    A. 其母知其諠也,呼而问之 \\
    B. 其母引刀裂其织,以此诫之 \\
    C. 意甚悦,固求市之 \\
    D. 自是之后,孟子不复諠矣
}
\paragraph{
7. “值一人对面而来,各不相让”中的“值”应解释为(\color{red}C\color{black})。 \\
    A. 当值 \\
    B. 值得 \\
    C. 适逢 \\
    D. 价值
}
\paragraph{
8. 以下并非冯梦龙作品的是(\color{red}D\color{black})。 \\
    A. 《喻世明言》 \\
    B. 《醒世恒言》 \\
    C. 《警世通言》 \\
    D. 《拍案惊奇》
}
\paragraph{
9. “汝姑持肉回陪客饭,待我与他对立在此” 中的“姑”应解释为(\color{red}B\color{black})。 \\
    A. 姑母 \\
    B. 暂且 \\
    C. 姑息 \\
    D. 婆婆
}
\paragraph{
10. 下列各句中的“食”字意义不同于其他三项的是(\color{red}C\color{black})。 \\
    A. 黔敖为食于路 \\
    B. 黔敖左奉食 \\
    C. 嗟!来食 \\
    D. 予唯不食嗟来之食
}
\paragraph{
11. “臣为王引弓虚发而下鸟”中,“下”字的用法是(\color{red}A\color{black})。 \\
    A. 使动用法 \\
    B. 意动用法 \\
    C. 为动用法 \\
    D. 形容词作动词
}
\paragraph{
12. 对“谓左右曰:‘吾甚恶紫之臭。’”中的词语解释错误的是(\color{red}D\color{black})。 \\
    A. 左右:侍从 \\
    B. 甚:很,非常 \\
    C. 恶:厌恶,讨厌 \\
    D. 臭:臭气
}
\paragraph{
13. 对“于是左右适有衣紫而进者”的“而”解释正确的是(\color{red}A\color{black})。 \\
    A. 连词,表示递进关系 \\
    B. 连词,表示修饰关系 \\
    C. 连词,表示并列关系 \\
    D. 连词,表示因果关系
}
\paragraph{
14. 以下并非表示“片刻,一会儿”的是(\color{red}D\color{black})。 \\
    A. 有间 \\
    B. 少时 \\
    C. 既而 \\
    D. 久之
}
\paragraph{
15. 与“今人不知以其愚心而师圣人之智,不亦过乎”的“过”含义相同的是(\color{red}C\color{black})。 \\
    A. 过故人庄 \\
    B. 从此道至吾军,不过二十里耳 \\
    C. 过而能改,善莫大焉 \\
    D. 三伏适已过,骄阳化为霖
}
\paragraph{
16. 庄子及其后学所著的《庄子》又名(\color{red}B\color{black})。 \\
    A. 《道德经》 \\
    B. 《南华经》 \\
    C. 《冲虚经》 \\
    D. 《通玄真经》
}
\paragraph{
17. 以下画线词注音不正确的是(\color{red}D\color{black})。 \\
    A. 宋人有善为不\uline{龟}手之药者(jūn) \\
    B. 今一朝而\uline{鬻}技百金(yù) \\
    C. 客得之,以\uline{说}吴王(shuì) \\
    D. 越有难,吴王使之\uline{将}(jiāng)
}
\paragraph{
18. 对“火不明,因谓持烛者曰……”中的“因”解释正确的是(\color{red}A\color{black})。 \\
    A. 于是 \\
    B. 因为 \\
    C. 照旧 \\
    D. 依据
}
\paragraph{
19. 以下对“战国四君子”表述有误的是(\color{red}B\color{black})。 \\
    A. 魏国的信陵君魏无忌 \\
    B. 齐国的孟尝君田忌 \\
    C. 赵国的平原君赵胜 \\
    D. 楚国的春申君黄歇
}
\paragraph{
20. 下列不属于“前四史”的是(\color{red}A\color{black})。 \\
    A. 《战国策》 \\
    B. 《史记》 \\
    C. 《汉书》 \\
    D. 《后汉书》
}
\paragraph{
21. 和“悫年少,问其所志”的“志”意思相同的是(\color{red}C\color{black})。 \\
    A. 吾十有五而志于学 \\
    B. 人之立志,故不如蜀鄙之僧哉 \\
    C. 燕雀安知鸿鹄之志哉 \\
    D. 便扶向路,处处志之
}
\paragraph{
22. 下列经典诗句并非李贺创作的一项是(\color{red}C\color{black})。 \\
    A. 衰兰送客咸阳道,天若有情天亦老 \\
    B. 男儿何不带吴钩,收取关山五十州 \\
    C. 此情可待成追忆,只是当时已惘然 \\
    D. 黑云压城城欲摧,甲光向日金鳞开
}
\paragraph{
23. 以下成语故事主人公不是晏子的是(\color{red}D\color{black})。 \\
    A. 比肩接踵 \\
    B. 南橘北枳 \\
    C. 二桃杀三士 \\
    D. 不入虎穴焉得虎子
}
\paragraph{
24. 下列句中画线的“以”解释为“认为,以为”的是(\color{red}C\color{black})。 \\
    A. 烛邹有罪三,请数之\uline{以}其罪而杀之 \\
    B. 使吾君\uline{以}鸟之故杀人,是罪二也 \\
    C. 使诸侯闻之,\uline{以}吾君重鸟以轻士 \\
    D. 姥闻羲之将至,烹\uline{以}待之
}
\paragraph{
25. 下列句中“然”字用法与其他三项不同的是(\color{red}B\color{black})。 \\
    A. 河内凶亦然 \\
    B. 填然鼓之,兵刃既接 \\
    C. 羲之欣然写毕,笼鹅而归 \\
    D. 有饿者蒙袂辑屦,贸贸然来
}
\paragraph{
26. 下面句子停顿正确的是(\color{red}B\color{black})。 \\
    A. 有道之/士怀其术/而欲明/万乘之主 \\
    B. 有道之士/怀其术/而欲明/万乘之主 \\
    C. 有道之/士怀其术而欲/明万乘之主 \\
    D. 有道之士/怀其术而欲/明/万乘之主
}
\paragraph{
27. 下列画线词解释错误的一项是(\color{red}C\color{black})。 \\
    A. \uline{为}酒甚美(做) \\
    B. \uline{怪}其故(对……感到奇怪) \\
    C. \uline{或}令孺子怀钱(有时) \\
    D. 挈壶瓮而\uline{往}酤(到那里去)
}
\paragraph{
28. 对“尝与其父奢言兵事,奢不能难,然不谓善”中字词解释不正确的是(\color{red}D\color{black})。 \\
    A. 尝:曾经 \\
    B. 兵事:用兵打仗的事 \\
    C. 难:驳倒 \\
    D. 善:善良
}
\paragraph{
29. 对下列句中的“将”字解释不正确的是(\color{red}C\color{black})。 \\
    A. 使赵不将括即已(使……为将) \\
    B. 及括将行(将要) \\
    C. 括不可使将(将军) \\
    D. 秦将白起闻之(将军)
}
\paragraph{
30. 与“魏武亦记之”中“之”的用法相同的是(\color{red}C\color{black})。 \\
    A. 本在冀州之南 \\
    B. 醉翁之意不在酒 \\
    C. 属予作文以记之 \\
    D. 吾欲之南海
}
\paragraph{
31. 对下列句中的“语”注音及释义不完全正确的是(\color{red}D\color{black})。 \\
    A. 此犬黠慧能解人语(yǔ,话) \\
    B. 因戏语犬曰(yù,对……说) \\
    C. 马上相逢无纸笔,凭君传语报平安(yǔ,话) \\
    D. 贫者语于富者曰(yǔ,告诉)
}
\paragraph{
32. 下列画线词与“黄耳传\uline{书}”的“书”字解释相同的是(\color{red}A\color{black})。 \\
    A. 烽火连三月,家\uline{书}抵万金 \\
    B. 大抵观\uline{书}先须熟读 \\
    C. 乃丹\uline{书}帛曰:陈胜王 \\
    D. 军\uline{书}十二卷,卷卷有爷名
}
\paragraph{
33. 下列句中的“颇” 的含义与其他三项不同的是(\color{red}B\color{black})。 \\
    A. 稍长,亦颇驯,竟忘其为狼 \\
    B. 家贫,颇蓄薄酿 \\
    C. 客请与予对局,予颇易之 \\
    D. 是日风静,舟行颇迟
}
\paragraph{
34. 《阅微草堂笔记》的作者是(\color{red}B\color{black})。 \\
    A. 沈括 \\
    B. 纪昀 \\
    C. 洪迈 \\
    D. 段成式
}
\paragraph{
35. 下列句中的“诸” 的含义与其他三项不同的是(\color{red}D\color{black})。 \\
    A. 工之侨以归,谋诸漆工 \\
    B. 匣而埋诸土,期年出之 \\
    C. 投诸渤海之尾 \\
    D. 诸峰之顶,亦低于山顶之地面
}
\paragraph{
36. 《后汉书》的作者是(\color{red}C\color{black})。 \\
    A. 东汉历史学家班固 \\
    B. 西汉历史学家班固 \\
    C. 南朝历史学家范晔 \\
    D. 宋代历史学家范晔
}
\paragraph{
37. 下列句中画线词的解释与当今解释不同的是(\color{red}C\color{black})。 \\
    A. 近\uline{学堂} \\
    B. 以雇\uline{书生}抄书 \\
    C. 令育与其子\uline{同学} \\
    D. 育遂博通\uline{经史}
}
\paragraph{
38. 下列画线词的词意与用法相同的一项是(\color{red}A\color{black})。 \\
    A. 育将鬻己\uline{以}偿 / 无从致书\uline{以}观 \\
    B. 其主笞\uline{之} / 吾欲\uline{之}南海 \\
    C. 郭子敬闻\uline{而}嘉之 / \uline{而}山不增 \\
    D. 令育与\uline{其}子同学 / 欲穷\uline{其}林
}
\paragraph{
39. 下面句子停顿正确的是(\color{red}A\color{black})。 \\
    A. 尤工秋岚/远景多写江南真山/不为奇峭之笔 \\
    B. 尤工秋岚远景多/写江南真山不为奇峭之笔 \\
    C. 尤工秋岚/远景多写江南/真山不为奇峭之笔 \\
    D. 尤工秋岚远景/多写江南真山/不为奇峭之笔
}
\paragraph{
40. 与“近视之几不类物象”的“几”解释相同的一项是(\color{red}C\color{black})。 \\
    A. 矗不知其几千万落 \\
    B. 从余问古事,或凭几学书 \\
    C. 今吾嗣为之十二年,几死者数矣 \\
    D. 子来几日矣
}
\paragraph{
41. 下列画线词与“今日不雨”中“雨”的用法不同的是(\color{red}D\color{black})。 \\
    A. 驴不胜怒,\uline{蹄}之 \\
    B. 一狼\uline{洞}其中 \\
    C. \uline{腰}白玉之环 \\
    D. \uline{箕畚}运于渤海之尾
}
\paragraph{
42. 下列句中的“所以” 的含义与其他三项不同的是(\color{red}A\color{black})。 \\
    A. 吾所以来此者,忆旧也 \\
    B. 法者国家所以布大信于天下也 \\
    C. 谷者,所以食也 \\
    D. 师者,所以传道也
}
\paragraph{
43. 下列成语典故,与刘邦无关的一项是(\color{red}C\color{black})。 \\
    A. 约法三章 \\
    B. 人为刀殂,我为鱼肉 \\
    C. 望梅止渴 \\
    D. 分我杯羹
}
\paragraph{
44. 下列句中“为”的用法和意义与其他三项不同是(\color{red}B\color{black})。 \\
    A. 子墨子解带为城 \\
    B. 夫为天下除残贼 \\
    C. 宜缟素为资 \\
    D. 武陵人捕鱼为业
}
\paragraph{
45. 对下列句中的“数”解释错误的是(\color{red}C\color{black})。 \\
    A. 数为边吏(屡次) \\
    B. 率不过数岁即复倍约(几) \\
    C. 请数之以其罪而杀之(考察) \\
    D. 数烛邹罪已毕,请杀之(历数,列举)
}
\paragraph{
46. 《竹似贤》一文采用了(\color{red}D\color{black})的写作手法。 \\
    A. 欲扬先抑 \\
    B. 借景抒情 \\
    C. 反面衬托 \\
    D. 托物言志
}
\paragraph{
47. “每于断林荒荆间,一再鼓之,凄禽寒鹘,相和悲鸣” 采用了(\color{red}B\color{black})的手法。 \\
    A. 侧面描写 \\
    B. 正面描写 \\
    C. 借物喻人 \\
    D. 动静结合
}
\paragraph{
48. 以下并非描写西湖的诗歌是(\color{red}D\color{black})。 \\
    A. 白居易《钱塘湖春行》 \\
    B. 苏轼《饮湖上初晴后雨》 \\
    C. 苏轼《六月二十七日望湖楼醉书》 \\
    D. 张志和《渔歌子》
}
\paragraph{
49. 《浮生六记》 是清代作家(\color{red}B\color{black})的名作。 \\
    A. 袁枚 \\
    B. 沈复 \\
    C. 洪升 \\
    D. 张岱
}
\paragraph{
50. “使之冲烟而飞鸣,作青云白鹤观”中的“作”应解释为(\color{red}B\color{black})。 \\
    A. 作为 \\
    B. 当作 \\
    C. 造成 \\
    D. 制造
}
\paragraph{
51. 下列对“故山谷尝自谓得草法于涪陵”翻译正确的一项是(\color{red}C\color{black})。 \\
    A. 所以山谷曾经在涪陵得意于悟到了写草书的方法 \\
    B. 所以山谷曾经得意于在涪陵悟到了写草书的方法 \\
    C. 所以山谷曾经在涪陵自己说悟到了写草书的方法 \\
    D. 所以山谷曾经自己说在涪陵悟到了写草书的方法
}
\paragraph{
52. 与“如履平地”中的“履”含义相同的是(\color{red}C\color{black})。 \\
    A. 削足适履 \\
    B. 西装革履 \\
    C. 履险如夷 \\
    D. 履行义务
}
\paragraph{
53. 对下列句中的“以为”解释不正确的是(\color{red}D\color{black})。 \\
    A. 以为且噬己也(认为) \\
    B. 虎见之,庞然大物也,以为神(把……当作) \\
    C. 明皇以李林甫为相(任用……做) \\
    D. 吾夫存日,以弹絮为业(让……做)
}
\paragraph{
54. 与“若任人不当”中“当”的读音和解释都相同的是(\color{red}D\color{black})。 \\
    A. 当窗理云鬃,对镜贴花黄 \\
    B. 募有能捕之者,当其租入 \\
    C. 当春乃发生 \\
    D. 臣不得越官而有功,不得陈言而不当
}
\paragraph{
55. 与“帝曲宴近臣于禁苑中”的“于”字用法不同的是(\color{red}C\color{black})。 \\
    A. 每于断林荒荆间,一再鼓之 \\
    B. 于厅事之东北角,施八尺屏障 \\
    C. 所欲有甚于生者,故不为苟得也 \\
    D. 太守与客来饮于此
}
\paragraph{
56. “屡迁至镇江都统制、扬州承宣使、骁卫上将军” 中的“迁”意为(\color{red}B\color{black})。 \\
    A. 贬谪 \\
    B. 调动官职 \\
    C. 变动 \\
    D. 更换
}
\paragraph{
57. “后以老病致仕,始居于霅”中的“致仕”意为(\color{red}A\color{black})。 \\
    A. 辞官 \\
    B. 任职 \\
    C. 贬谪 \\
    D. 升职
}
\paragraph{
58. 下面句子停顿正确的是(\color{red}B\color{black})。 \\
    A. 马耸耳以听/汪然出涕喑哑长鸣数声/而毙 \\
    B. 马耸耳以听/汪然出涕/喑哑长鸣数声而毙 \\
    C. 马耸耳/以听汪然/出涕喑哑长鸣数声而毙 \\
    D. 马耸耳以听/汪然出涕喑哑长鸣/数声而毙
}
\paragraph{
59. 下列句中的“何”与“莲叶何田田”的“何”意义相同的是(\color{red}D\color{black})。 \\
    A. 以此攻城,何城不克 \\
    B. 而其问何下而恭也 \\
    C. 肉食者谋之,又何间焉 \\
    D. 何乐而不为
}
\paragraph{
60. “亭亭山上松”的“亭亭”意为(\color{red}D\color{black})。 \\
    A. 十分秀气的样子 \\
    B. 明亮美好的样子 \\
    C. 高贵威严的样子 \\
    D. 高耸直立的样子
}
\paragraph{
61. “劝君惜取少年时”的“取”,正确的解释应是(\color{red}C\color{black})。 \\
    A. 得到,取得 \\
    B. 拿 \\
    C. 助词,用在动词后,无实在的意义 \\
    D. 采用,选取
}
\paragraph{
62. “却下水晶帘,玲珑望秋月”中月亮意象代表的意义是(\color{red}D\color{black})。 \\
    A. 凄清和孤苦 \\
    B. 寄托怀乡之情 \\
    C. 跨越时空的见证者 \\
    D. 代表恋情与相思
}
\paragraph{
63. 以下描写寒食节的诗句是(\color{red}D\color{black})。 \\
    A. 月上柳梢头,人约黄昏后 \\
    B. 屈子冤魂终古在,楚乡遗俗至今留 \\
    C. 但愿人长久,千里共婵娟 \\
    D. 芳洲拾翠暮忘归,秀野踏青来不定
}
\paragraph{
64. 以下对《闺意献张水部》理解错误的是(\color{red}D\color{black})。 \\
    A. 诗人把自己比作新娘 \\
    B. 把张水部比作新郎 \\
    C. 把主考官比作舅姑(公婆) \\
    D. 其写作意图是询问张水部是否欣赏自己的作品
}
\paragraph{
65. 以下不是田园诗代表人物的是(\color{red}B\color{black})。 \\
    A. 王维 \\
    B. 王昌龄 \\
    C. 孟浩然 \\
    D. 范成大
}
\paragraph{
66. 北宋文学家黄庭坚是(\color{red}B\color{black})诗派的开山之祖。 \\
    A. 山西 \\
    B. 江西 \\
    C. 陕西 \\
    D. 浙西
}
\paragraph{
67. 对“微与!其嗟也可去,其谢也可食”一句中字词解释正确的是(\color{red}C\color{black})。 \\
    A. 微:稍微 \\
    B. 其:那 \\
    C. 去:离开 \\
    D. 食:食物
}
\paragraph{
68. 对“闻弦音引而高飞,故疮陨也”一句中字词解释错误的是(\color{red}B\color{black})。 \\
    A. 闻:听到 \\
    B. 引:展翅 \\
    C. 故:所以 \\
    D. 陨:掉落
}
\paragraph{
69. 对“雨雪三日而不霁。公被狐白之裘,坐于堂侧陛”一句中字词注音错误的是(\color{red}C\color{black})。 \\
    A. 雨(yù) \\
    B. 霁(jì) \\
    C. 被(bèi) \\
    D. 陛(bì)
}
\paragraph{
70. 对“夫为天下者,亦奚以异乎牧马者哉” 一句中字词解释正确的是(\color{red}A\color{black})。 \\
    A. 为:治理 \\
    B. 奚:奚落 \\
    C. 异:奇怪 \\
    D. 乎:语气词
}
\paragraph{
71. 对“王杀无罪之臣,而明人之欺王也”一句翻译正确的是(\color{red}D\color{black})。 \\
    A. 大王杀死无罪的臣子,就明白有人在欺骗大王 \\
    B. 大王杀死无罪的臣子,明天就有人欺骗大王 \\
    C. 大王杀死无罪的臣子,明明是有人在欺骗大王 \\
    D. 大王杀死无罪的臣子,证明有人在欺骗大王
}
\paragraph{
72. 对“郢人有遗燕相国书者”一句中字词注音不正确的是(\color{red}D\color{black})。 \\
    A. 郢(yǐng) \\
    B. 遗(wèi) \\
    C. 燕(yān) \\
    D. 相(xiāng)
}
\paragraph{
73. 对“皆下士喜宾客以相倾”一句翻译正确的是(\color{red}B\color{black})。 \\
    A. (所以)都礼贤下士,结交宾客,(以此)来相互结交 \\
    B. (他们)都礼贤下士,结交宾客,(以此)来相互竞争 \\
    C. (他们)都(尽力)让下士、宾客开心,(以此)来相互结交 \\
    D. (他们)都(尽力)让下士、宾客开心,(以此)来相互竞争
}
\paragraph{
74. 对“主人嘿然不应” 一句中字词注音、解释均正确的是(\color{red}B\color{black})。 \\
    A. 嘿(mò),应(yīng) \\
    B. 嘿(mò),应(yìng) \\
    C. 嘿(hēi),应(yīng) \\
    D. 嘿(hēi),应(yìng)
}
\paragraph{
75. 对《江南》中“鱼戏莲叶间”一句的理解,错误的一项是(\color{red}C\color{black})。 \\
    A. 描写了鱼儿在莲叶之间嬉戏玩耍的场景 \\
    B. “戏”字写出了鱼儿自由、欢快的特点 \\
    C. 从侧面衬托水之清澈,表达诗人对池塘的喜爱之情 \\
    D. 在整首诗歌中起着承上启下的作用,使上下相连
}
\paragraph{
76. “风声一何\_\_\_\_,松枝一何\_\_\_\_”,空格处应填入的是( \color{red}D\color{black}  )。 \\
    A. 紧;正 \\
    B. 劲;盛 \\
    C. 紧;劲 \\
    D. 盛;劲
}
\paragraph{
77. 对“玉阶\uline{生}白露,夜久\uline{侵}罗袜”中画线词解释正确的是(\color{red}D\color{black})。 \\
    A. 生长;浸湿 \\
    B. 滋生;侵占 \\
    C. 生长;侵犯 \\
    D. 滋生;浸湿
}
\paragraph{
78. 韩翃的《寒食》是借\_\_\_\_(朝代)之事对\_\_\_\_(朝代)的社会现实进行批判。正确的选项是( \color{red}B\color{black} )。 \\
    A. 秦;唐 \\
    B. 汉;唐 \\
    C. 三国;唐 \\
    D. 隋;唐
}
\paragraph{
79. “洞房昨夜\_\_\_\_红烛,待晓堂前\_\_\_\_舅姑”,空格处应填入的是(\color{red}C\color{black})。 \\
    A. 燃;待 \\
    B. 举;问 \\
    C. 停;拜 \\
    D. 留;谢
}
\paragraph{
80. 对“更深月色半人家,北斗阑干南斗斜”中词语解释错误的是( \color{red}C\color{black} )。 \\
    A. 更:古时夜间计时单位 \\
    B. 北斗:即北斗星 \\
    C. 阑干:即栏杆 \\
    D. 南斗:星宿名
}
\paragraph{
81. 与刘方平《月夜》所写的体验有异曲同工之妙的诗句是( \color{red}B\color{black} )。 \\
    A. 竹外桃花三两枝,春江水暖鸭先知 \\
    B. 乱花渐欲迷人眼,浅草才能没马蹄 \\
    C. 等闲识得东风面,万紫千红总是春 \\
    D. 小荷才露尖尖角,早有蜻蜓立上头
}
\paragraph{
82. 下面的边塞诗名句与作者匹配不正确的是( \color{red}D\color{black} )。 \\
    A. 大漠孤烟直,长河落日圆——王维 \\
    B. 忽如一夜春风来,千树万树梨花开——岑参 \\
    C. 大漠沙如雪,燕山月似钩——李贺 \\
    D. 报君黄金台上意,提携玉龙为君死——李益
}
\paragraph{
83. “别梦依依到谢家”中的“谢家”,是指( \color{red}B\color{black} )。 \\
    A. 作者一个姓谢的朋友家 \\
    B. 泛指闺中女子居住之所 \\
    C. 谢道韫的家 \\
    D. 一个姓谢的女子的家
}
\paragraph{
84. 对“劳歌一曲解行舟” 中,“劳歌”解释错误的是( \color{red}A\color{black} )。 \\
    A. 拉纤的船夫唱的歌曲 \\
    B. 原指在劳劳亭送客时唱的歌 \\
    C. 送别歌的代称 \\
    D. 与“旅人倚征棹,薄暮起劳歌”中的“劳歌”意思相同
}
\paragraph{
85. 以下对李贺的介绍不正确的是( \color{red}C\color{black} )。 \\
    A. 后世称李昌谷 \\
    B. 有“诗鬼”之称 \\
    C. 与杜牧并称“小李杜” \\
    D. 与李白、李商隐并称唐代“三李”
}
\paragraph{
86. 凌烟阁是( \color{red}B\color{black} )为表彰功臣而建的殿阁。 \\
    A. 汉武帝 \\
    B. 唐太宗 \\
    C. 汉光武帝 \\
    D. 宋理宗
}
\paragraph{
87. 《寄李儋元锡》一诗的作者是( \color{red}D\color{black} )。 \\
    A. 白居易 \\
    B. 元稹 \\
    C. 李商隐 \\
    D. 韦应物
}
\paragraph{
88. 对以下文言名句的比喻义理解错误的是( \color{red}C\color{black} )。 \\
    A. 七年之病,求三年之艾——比喻要事先做准备 \\
    B. 见卵而求时夜——比喻言之过早 \\
    C. 甘瓜苦蒂——比喻要成功必须要吃苦 \\
    D. 成也萧何,败也萧何——比喻出尔反尔
}
\paragraph{
89. “三过其门而不入”的典故与( \color{red}C\color{black} )有关。 \\
    A. 尧 \\
    B. 舜 \\
    C. 禹 \\
    D. 汤
}
\paragraph{
90. “三年不窥园” 的典故与( \color{red}A\color{black} )有关。 \\
    A. 董仲舒 \\
    B. 李密 \\
    C. 匡衡 \\
    D. 车胤
}
\paragraph{
91. “山雨欲来风满楼”出自( \color{red}B\color{black} )的诗句。 \\
    A. 李益 \\
    B. 许浑 \\
    C. 苏轼 \\
    D. 王建
}
\paragraph{
92. “五色令人目盲,五音令人耳聋”出自( \color{red}D\color{black} )。 \\
    A. 《论语》 \\
    B. 《孟子》 \\
    C. 《南华经》 \\
    D. 《道德经》
}
\paragraph{
93. “见善如不及,见不善如\_\_\_\_。”空格处应填入( \color{red}B\color{black} )。 \\
    A. 不闻 \\
    B. 探汤 \\
    C. 世仇 \\
    D. 猛兽
}
\paragraph{
94. “为他人作嫁衣裳” 出自( \color{red}A\color{black} )的诗句。 \\
    A. 秦韬玉 \\
    B. 白居易 \\
    C. 韦庄 \\
    D. 李商隐
}
\paragraph{
95. 对“苍蝇附骥尾而致千里”理解有误的是( \color{red}D\color{black} )。 \\
    A. “骥”指千里马 \\
    B. 意为:苍蝇依附在千里马的尾巴上才到达千里之外 \\
    C. 比喻普通的人因为依附于有学问有道德的人而成名 \\
    D. 也作敬称,简化为“附骥”或“附骥尾”
}
\paragraph{
96. “掷地作金石声” 不能用来形容( \color{red}C\color{black} )。 \\
    A. 文章词藻华美,语调铿锵 \\
    B. 人才华之高 \\
    C. 器物制作精良 \\
    D. 落实力度大
}
\paragraph{
97. 对“解衣衣我,推食食我”解释有误的是( \color{red}B\color{black} )。 \\
    A. 出自《史记·淮阴侯列传》 \\
    B. 第二个“衣”和“食”都是名词 \\
    C. 形容对自己的热切关怀 \\
    D. 表示关系很密切
}
\paragraph{
98. 对“食不厌精,脍不厌细”解释有误的是( \color{red}A\color{black} )。 \\
    A. 出自《孟子》 \\
    B. “厌”是“满足”的意思 \\
    C. “精“指”上等好米。“脍”指切细的鱼或肉 \\
    D. 形容饮食十分讲究
}
\paragraph{
99. 对“别而听之则愚,合而听之则圣”解释有误的是( \color{red}C\color{black} )。 \\
    A. 出自《管子》 \\
    B. “别”指个别。“合”指全面 \\
    C. “圣”是“圣人”的意思 \\
    D. 意为不能偏听偏信,要全面听取意见
}
\paragraph{
100. 对“翻手为云覆手雨”解释有误的是( \color{red}D\color{black} )。 \\
    A. 出自杜甫诗句 \\
    B. 意为:把手朝天一翻就是云,把手掌倒过来就是雨 \\
    C. 形容人情变化迅速 \\
    D. 也形容人不能控制自己的情绪
}
\paragraph{
101. 典故“过鲤庭”喻指的是( \color{red}D\color{black} )。 \\
    A. 君子不偏爱自己的儿子 \\
    B. 与长辈交谈要守礼 \\
    C. 学习要举一反三 \\
    D. 晚辈接受老师、家长的教育
}
\paragraph{
102. 被王国维称为“词中老杜”的词人是( \color{red}C\color{black} )。 \\
    A. 欧阳修 \\
    B. 范仲淹 \\
    C. 周邦彦 \\
    D. 王安石
}
\paragraph{
103. 古代在科举考试后,州县长官常常举办“(\color{red}B\color{black})”,来表达对人才的尊重和庆贺。 \\
    A. 文会宴 \\
    B. 鹿鸣宴 \\
    C. 孔府宴 \\
    D. 八珍宴
}
\paragraph{
104. 被评价为“文采若云月,高深若山海”的典籍是( \color{red}D\color{black} )。 \\
    A. 《尚书》 \\
    B. 《庄子》 \\
    C. 《史记》 \\
    D. 《左传》
}
\paragraph{
105. 《水浒传》里朱贵的绰号是“旱地忽律”,“忽律”指的是(\color{red}A\color{black}  )。 \\
    A. 鳄鱼 \\
    B. 蛟龙 \\
    C. 青蛙 \\
    D. 鲤鱼
}
\paragraph{
106. 古代人们常常在绘画中借用白鹭表达良好的愿望,以下表述不正确的是( \color{red}C\color{black} )。 \\
    A. 白鹭、莲花和荷叶表示“一路连科” \\
    B. 白鹭与芙蓉花寓意“一路荣华” \\
    C. 白鹭与猴子寓意“一路封侯” \\
    D. 白鹭与牡丹寓意“一路富贵”
}
\paragraph{
107. “黄鸟为悲鸣,哀哉伤肺肝。”(曹植《三良诗》)“黄鸟悲鸣”的典故用来( \color{red}A\color{black} )。 \\
    A. 悲悼冤死的贤才 \\
    B. 怀念逝去的亲人 \\
    C. 表达心中的悲哀 \\
    D. 哀叹岁月的流逝
}
\paragraph{
108. 以下词句中描写文化名城南京的是( \color{red}B\color{black} )。 \\
    A. 重湖叠巘清嘉,有三秋桂子,十里荷花 \\
    B. 彩舟云淡,星河鹭起,画图难足 \\
    C. 斜阳草树,寻常巷陌,人道寄奴曾住 \\
    D. 长忆观潮,满郭人争江上望
}
\paragraph{
109. 以下并非颜真卿书法作品的是( \color{red}B\color{black} )。 \\
    A. 《颜勤礼碑》 \\
    B. 《九成宫醴泉铭》 \\
    C. 《多宝塔碑》 \\
    D. 《祭侄文稿》
}
\paragraph{
110. “西陆蝉声唱,南冠客思侵。”(骆宾王《在狱咏蝉》)“南冠”指( \color{red}C\color{black} )。 \\
    A. 旅人 \\
    B. 宾客 \\
    C. 囚徒 \\
    D. 诗人
}
\paragraph{
111. “傲世曾歌楚人凤,著书久绝鲁郊麟。”诗句中使用了“( \color{red}D\color{black} )悲麟”的典故。 \\
    A. 庄子 \\
    B. 孟子 \\
    C. 老子 \\
    D. 孔子
}
\paragraph{
112. 咏物诗“眇形才脱粪中胎,鼓翅摇头可恶哉。苦不自量何种类,玉阶金殿也飞来”写的是( \color{red}C\color{black} )。 \\
    A. 蜣螂 \\
    B. 蚊子 \\
    C. 苍蝇 \\
    D. 金龟子
}
\paragraph{
113. “每逢佳节倍思亲”中的“佳节”是指( \color{red}D\color{black} )。 \\
    A. 清明节 \\
    B. 乞巧节 \\
    C. 中秋节 \\
    D. 重阳节
}
\paragraph{
114. 下列属于李清照词集的是( \color{red}A\color{black} )。 \\
    A. 《漱玉集》 \\
    B. 《断肠集》 \\
    C. 《饮水词》 \\
    D. 《稼轩长短句》
}
\paragraph{
115. “尽信《书》,则不如无《书》”这一名言的作者是( \color{red}B\color{black} )。 \\
    A. 庄子 \\
    B. 孟子 \\
    C. 孔子 \\
    D. 韩非子
}
\paragraph{
116. 下面不属于古诗中月亮的美称的是( \color{red}C\color{black} )。 \\
    A. 广寒宫 \\
    B. 冰镜 \\
    C. 弯弓 \\
    D. 白玉盘
}
\paragraph{
117. 与《木兰诗》合称“乐府双璧”的是( \color{red}D\color{black} )。 \\
    A. 《陌上桑》 \\
    B. 《江南》 \\
    C. 《陇西行》 \\
    D. 《孔雀东南飞》
}
\paragraph{
118. 对成语“美轮美奂”中“轮”的解释正确的一项是( \color{red}B\color{black} )。 \\
    A. 圆润 \\
    B. 高大 \\
    C. 车轮 \\
    D. 玉盘
}
\paragraph{
119. 成语“匹夫之勇”中“匹夫”是指( \color{red}C\color{black} )。 \\
    A. 武夫 \\
    B. 勇士 \\
    C. 一个人 \\
    D. 大丈夫
}
\paragraph{
120. 下面句中的“百”不是表示约数的是( \color{red}D\color{black} )。 \\
    A. 百闻不如一见 \\
    B. 百口莫辩 \\
    C. 虑祸百之 \\
    D. 齐宣王使人吹竽,必三百人
}
\paragraph{
121. 古代“六艺”“礼、乐、射、御、书、数”中的“御”是指( \color{red}C\color{black} )。 \\
    A. 下棋 \\
    B. 武术 \\
    C. 驾车 \\
    D. 防守
}
\paragraph{
122. “一寸光阴一寸金。”量出“一寸”时间单位的古代计时仪器是( \color{red}A\color{black} )。 \\
    A. 圭表 \\
    B. 日晷 \\
    C. 漏刻 \\
    D. 漏壶
}
\paragraph{
123. 下列地支与生肖配对正确的是( \color{red}B\color{black} )。 \\
    A. 子——兔 \\
    B. 巳——蛇 \\
    C. 酉——猴 \\
    D. 申——鸡
}
\paragraph{
124. “所谓华山洞者,以其乃华山之阳名之也。”据此句可知华山洞在华山的( \color{red}B\color{black} )。 \\
    A. 东面 \\
    B. 南面 \\
    C. 西面 \\
    D. 北面
}
\paragraph{
125. “孟春”是指阴历( \color{red}A\color{black} )。 \\
    A. 正月 \\
    B. 二月 \\
    C. 三月 \\
    D. 四月
}
\paragraph{
126. “蓬莱文章建安骨”,“建安骨”这一文学风格形成的时期是( \color{red}B\color{black} )。 \\
    A. 魏晋 \\
    B. 汉末 \\
    C. 唐代 \\
    D. 五代
}
\paragraph{
127. “入木三分”这个典故的原意是用来形容( \color{red}C\color{black} )。 \\
    A. 雕刻技术精湛 \\
    B. 文章内容深刻 \\
    C. 书法笔力强劲 \\
    D. 射箭本领高超
}
\paragraph{
128. 下列表客套的词语中,解说不当的一项是( \color{red}C\color{black} )。 \\
    A. 祝贺喜事说“恭喜” \\
    B. 请人予以方便说“借光” \\
    C. 慰问别人说“劳驾” \\
    D. 请人批评说“指教”
}
\paragraph{
129. 古人多给自己取号“某某居士”,下列居士对应错误的一组是( \color{red}B\color{black} )。 \\
    A. 香山居士——白居易 \\
    B. 茶山居士——杨万里 \\
    C. 易安居士——李清照 \\
    D. 石湖居士——范成大
}
\paragraph{
130. 与下列成语相关的人物配对错误的一项是( \color{red}C\color{black} )。 \\
    A. 乘风破浪——宗悫 \\
    B. 击楫中流——祖逖 \\
    C. 囊萤映雪——孙敬 \\
    D. 指鹿为马——赵高
}
\paragraph{
131. 下列诗句不是出自清代郑燮笔下的是( \color{red}B\color{black} )。 \\
    A. 千磨万击还坚劲,任尔东西南北风 \\
    B. 粉骨碎身浑不怕,要留清白在人间 \\
    C. 些小吾曹州县吏,一枝一叶总关情 \\
    D. 新竹高于旧竹枝,全凭老干为扶持
}
\paragraph{
132. “道是春来花未,道是雪来香异。竹外一枝斜,野人家。”词句描写的是( \color{red}A\color{black} )。 \\
    A. 梅花 \\
    B. 杏花 \\
    C. 桃花 \\
    D. 梨花
}
\paragraph{
133. 下列诗句中没有运用“动静相衬”手法的是( \color{red}B\color{black} )。 \\
    A. 鸟宿池边树,僧敲月下门 \\
    B. 千山鸟飞绝,万径人踪灭 \\
    C. 泉声咽危石,日色冷青松 \\
    D. 鸡声茅店月,人迹板桥霜
}
\paragraph{
134. “百岁光阴一梦蝶,重回首往事堪嗟。”“梦蝶”典故出自( \color{red}D\color{black} )。 \\
    A. 《文子》 \\
    B. 《列子》 \\
    C. 《老子》 \\
    D. 《庄子》
}
\paragraph{
135. 下列属于二十四节气的是( \color{red}B\color{black} )。 \\
    A. 元宵 \\
    B. 清明 \\
    C. 端午 \\
    D. 中秋
}
\paragraph{
136. 被后人称为“七绝圣手”的是( \color{red}A\color{black} )。 \\
    A. 王昌龄 \\
    B. 王之涣 \\
    C. 王维 \\
    D. 王勃
}
\paragraph{
137. 下列不属于宋代四大书院的是( \color{red}D\color{black} )。 \\
    A. 江西的白鹿洞书院 \\
    B. 湖南的岳麓书院 \\
    C. 河南的应天府书院 \\
    D. 江苏的东林书院
}
\paragraph{
138. 下列并称没有根据的一项是( \color{red}D\color{black} )。 \\
    A. 元白 \\
    B. 苏辛 \\
    C. 韩柳 \\
    D. 李孟
}
\paragraph{
139. 下列没有“贬职”含义的一项是( \color{red}A\color{black} )。 \\
    A. 擢 \\
    B. 迁 \\
    C. 谪 \\
    D. 黜
}
\paragraph{
140. 《说文解字》的编撰者是( \color{red}B\color{black} )代的许慎。 \\
    A. 秦 \\
    B. 汉 \\
    C. 晋 \\
    D. 隋
}
\paragraph{
141. 以下不属于古代酒器的是( \color{red}C\color{black} )。 \\
    A. 钟 \\
    B. 斗 \\
    C. 俎 \\
    D. 卮
}
\paragraph{
142. 下列成语注音有误的是( \color{red}A\color{black} )。 \\
    A. 放荡不\uline{羁}(jì) \\
    B. 分道扬\uline{镳}(biāo) \\
    C. 风雨如\uline{晦}(hùi) \\
    D. 百折不\uline{挠}(náo)
}
\paragraph{
143. 下列成语书写有误的是( \color{red}C\color{black} )。 \\
    A. 白璧微瑕 \\
    B. 耳濡目染 \\
    C. 众口烁金 \\
    D. 再接再厉
}
\paragraph{
144. 下列成语中画线词解释有误的一项是( \color{red}D\color{black} )。 \\
    A. 不可\uline{胜}数(尽) \\
    B. 兵不\uline{厌}诈(讨厌) \\
    C. 不\uline{爽}毫发(差错) \\
    D. \uline{掉}以轻心(摆弄)
}
\paragraph{
145. 因词中写过三句带“影”的名句而有“张三影”之称的是( \color{red}B\color{black} )。 \\
    A. 张藉 \\
    B. 张先 \\
    C. 张九龄 \\
    D. 张志和
}
\paragraph{
146. “老夫聊发少年狂”中的“老夫”指( \color{red}A\color{black} )。 \\
    A. 苏轼 \\
    B. 辛弃疾 \\
    C. 岳飞 \\
    D. 苏辙
}
\paragraph{
147. 下列不属于杜甫作品的是( \color{red}D\color{black} )。 \\
    A. 《望岳》 \\
    B. 《春夜喜雨》 \\
    C. 《登岳阳楼》 \\
    D. 《观猎》
}
\paragraph{
148. 人称“梅妻鹤子”的北宋诗人是( \color{red}D\color{black} )。 \\
    A. 张先 \\
    B. 杨万里 \\
    C. 周邦彦 \\
    D. 林逋
}
\paragraph{
149. 在下列古地别称中,指今天成都市的是( \color{red}D\color{black} )。 \\
    A. 金陵 \\
    B. 广陵 \\
    C. 姑苏 \\
    D. 锦官城
}
\paragraph{
150. 以下不属于古代“五行”的是( \color{red}C\color{black} )。 \\
    A. 金 \\
    B. 土 \\
    C. 气 \\
    D. 水
}
\newpage
\section{填空题}
\paragraph{1. 江南可\uline{\quad \textcolor{red}{采}\textcolor{red}{莲} \quad ,莲叶何田田。}}
\paragraph{2. 岂不罹\uline{\quad \textcolor{red}{凝}\textcolor{red}{寒} \quad ,松柏有本性。}}
\paragraph{3. 劝君莫惜\uline{\quad \textcolor{red}{金}\textcolor{red}{缕}\textcolor{red}{衣} \quad ,劝君惜取少年时。}}
\paragraph{4. 白头宫女在,闲坐说\uline{\quad \textcolor{red}{玄}\textcolor{red}{宗} \quad 。}}
\paragraph{5. 春城无处不飞花,寒食东风\uline{\quad \textcolor{red}{御}\textcolor{red}{柳} \quad 斜。}}
\paragraph{6. 妆罢低声问夫婿,画眉\uline{\quad \textcolor{red}{深}\textcolor{red}{浅} \quad 入时无?}}
\paragraph{7. 今夜偏知春气暖,虫声\uline{\quad \textcolor{red}{新}\textcolor{red}{透} \quad 绿窗纱。}}
\paragraph{8. 可怜\uline{\quad \textcolor{red}{无}\textcolor{red}{定}\textcolor{red}{河} \quad 边骨,犹是春闺梦里人。}}
\paragraph{9. 多情只有\uline{\quad \textcolor{red}{春}\textcolor{red}{庭}\textcolor{red}{月} \quad ,犹为离人照落花。}}
\paragraph{10. 日暮酒醒人已远,满天风雨\uline{\quad \textcolor{red}{下}\textcolor{red}{西}\textcolor{red}{楼} \quad 。}}
\paragraph{11. 不知近水花先发,疑是\uline{\quad \textcolor{red}{经}\textcolor{red}{冬} \quad 雪未销。}}
\paragraph{12. 蜡烛有心还惜别,替人\uline{\quad \textcolor{red}{垂}\textcolor{red}{泪} \quad 到天明。}}
\paragraph{13. 男儿何不带\uline{\quad \textcolor{red}{吴}\textcolor{red}{钩} \quad ,收取关山五十州。}}
\paragraph{14. 未收天子\uline{\quad \textcolor{red}{河}\textcolor{red}{湟}\textcolor{red}{地} \quad ,不拟回头望故乡。}}
\paragraph{15. 借问梅花何处落,风吹一夜满\uline{\quad \textcolor{red}{关}\textcolor{red}{山} \quad 。}}
\paragraph{16. 风鸣两岸叶,月照一\uline{\quad \textcolor{red}{孤}\textcolor{red}{舟} \quad  。}}
\paragraph{17. 凉风起\uline{\quad \textcolor{red}{天}\textcolor{red}{末} \quad ,君子意如何?}}
\paragraph{18. 世事茫茫难自料,春愁\uline{\quad \textcolor{red}{黯}\textcolor{red}{黯} \quad 独成眠。}}
\paragraph{19. 卷地风来忽吹散,望湖楼下\uline{\quad \textcolor{red}{水}\textcolor{red}{如}\textcolor{red}{天} \quad  。}}
\paragraph{20. 凌霄不屈己,得地本\uline{\quad \textcolor{red}{虚}\textcolor{red}{心} \quad。}}
\paragraph{21. 遗民泪尽\uline{\quad \textcolor{red}{胡}\textcolor{red}{尘} \quad 里,南望王师又一年。}}
\paragraph{22. 江山代有才人出,各领\uline{\quad \textcolor{red}{风}\textcolor{red}{骚} \quad 数百年。}}
\paragraph{23. 西北望长安,可怜\uline{\quad \textcolor{red}{无}\textcolor{red}{数}\textcolor{red}{山} \quad 。}}
\paragraph{24. 海棠开后,梨花暮雨,\uline{\quad \textcolor{red}{燕}\textcolor{red}{子} \quad 空楼。}}
\paragraph{25. 人之将死,\uline{\quad \textcolor{red}{其}\textcolor{red}{言} \quad 也善。}}
\paragraph{26. 万事俱备,只欠\uline{\quad \textcolor{red}{东}\textcolor{red}{风} \quad。}}
\paragraph{27. 千里之堤,\uline{\quad \textcolor{red}{溃}\textcolor{red}{于} \quad 蚁穴。}}
\paragraph{28. 瓜田不纳履,李下不\uline{\quad \textcolor{red}{整}\textcolor{red}{冠} \quad。}}
\paragraph{29. 司马昭之心,\uline{\quad \textcolor{red}{路}\textcolor{red}{人}\textcolor{red}{皆}\textcolor{red}{知} \quad。}}
\paragraph{30. 只许州官放火,不许\uline{\quad \textcolor{red}{百}\textcolor{red}{姓}\textcolor{red}{点}\textcolor{red}{灯} \quad。}}
\paragraph{31. 兼听则明,\uline{\quad \textcolor{red}{偏}\textcolor{red}{信}\textcolor{red}{则}\textcolor{red}{暗} \quad。}}
\paragraph{32. 工欲\uline{\quad \textcolor{red}{善}\textcolor{red}{其}\textcolor{red}{事} \quad ,必先利其器。}}
\paragraph{33. \uline{\quad \textcolor{red}{上}\textcolor{red}{有}\textcolor{red}{所}\textcolor{red}{好} \quad ,下必甚焉。}}
\paragraph{34. 己所不欲,\uline{\quad \textcolor{red}{勿}\textcolor{red}{施}\textcolor{red}{于}\textcolor{red}{人} \quad 。}}
\paragraph{35. 家有\uline{\quad \textcolor{red}{弊}\textcolor{red}{帚} \quad ,享之千金。}}
\paragraph{36. 久旱\uline{\quad \textcolor{red}{逢}\textcolor{red}{甘}\textcolor{red}{霖} \quad,他乡遇故知。}}
\paragraph{37. 蓬生麻中,\uline{\quad \textcolor{red}{不}\textcolor{red}{扶} \quad 而直。}}
\paragraph{38. 人无 \uline{\quad \textcolor{red}{远}\textcolor{red}{虑} \quad ,必有近忧。}}
\paragraph{39. \uline{\quad \textcolor{red}{物}\textcolor{red}{以}\textcolor{red}{类}\textcolor{red}{聚} \quad,人以群分。}}
\paragraph{40. 天有不测风云,人有\uline{\quad \textcolor{red}{悲}\textcolor{red}{欢}\textcolor{red}{离}\textcolor{red}{合} \quad。}}
\paragraph{41. 《后汉书》是一部记载东汉时期历史的 \uline{\quad \textcolor{red}{纪}\textcolor{red}{传} \quad 体断代史。}}
\paragraph{42. 中国纪传体史书的体裁之一\uline{\quad \textcolor{red}{列}\textcolor{red}{传} \quad ,是帝王诸侯外其他各方面代表人物的生平事迹和少数民族的传记。}}
\paragraph{43. 黄庭坚的诗,被苏轼称为“\uline{\quad \textcolor{red}{山}\textcolor{red}{谷} \quad 体”。}}
\paragraph{44. 六经,是指经过\uline{\quad \textcolor{red}{孔}\textcolor{red}{子} \quad 整理而传授的六部先秦古籍。}}
\paragraph{45. 元稹与白居易为终生诗友,共创“\uline{\quad \textcolor{red}{元}\textcolor{red}{和} \quad 体”,并称“元白”。}}
\paragraph{46. 辛弃疾与\uline{\quad \textcolor{red}{李}\textcolor{red}{清}\textcolor{red}{照} \quad 并称“济南二安”。}}
\paragraph{47. 杜牧字牧之,号\uline{\quad \textcolor{red}{樊}\textcolor{red}{川} \quad 居士。}}
\paragraph{48. 高适与岑参、王昌龄、\uline{\quad \textcolor{red}{王}\textcolor{red}{之}\textcolor{red}{涣} \quad 合称“边塞四诗人”。}}
\paragraph{49. 唐代诗人孟浩然因籍贯而被世人称为“孟\uline{\quad \textcolor{red}{襄}\textcolor{red}{阳} \quad”。}}
\paragraph{50. 元代著名的散曲家、剧作家张可久与\uline{\quad \textcolor{red}{乔}\textcolor{red}{吉} \quad 并称“双璧”。}}
\label{lastpage}
\end{document}